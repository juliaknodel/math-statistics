\subsection{Распределения}
	\begin{itemize}
		\item Нормальное распределение \begin{equation}
										  N(x, 0, 1) = \frac{1}{\sqrt{2\pi}}e^{\frac{-x^2}{2}} \label{norm} \eqno(1)
									   \end{equation}
		\item Распределение Коши \begin{equation}
									C(x, 0, 1) = \frac{1}{\pi}\frac{1}{x^2+1} \label{koshi} \eqno(2)
								 \end{equation} 
		\item Распределение Лапласа \begin{equation}
									   L(x, 0, \frac{1}{\sqrt{2}}) = \frac{1}{\sqrt{2}}e^{-\sqrt{2}|x|} \label{laplace} \eqno(3)
									\end{equation}
		\item Распределение Пуассона \begin{equation}
										P(k, 10) = \frac{10^k}{k!}e^{-10}\label{puasson} \eqno(4)
									 \end{equation}
		\item Равномерное распределение \begin{equation}
				U(x, -\sqrt{3}, \sqrt{3}) =
				\begin{cases}
					\frac{1}{2\sqrt{3}} &\text{$при |x|\leq \sqrt{3}$}\\
					0 &\text{$при |x|>\sqrt{3}$}
				\end{cases}
				\label{uni} \eqno(5)
			\end{equation}
	\end{itemize}

	\subsection{Гистограмма}
	\subsubsection{Определение}
	\textit{Гистограмма} в математической статистике — это функция, приближающая плотность вероятности некоторого распределения, построенная на основе выборки из него.
	
	\subsubsection{Графическое описание}
	Графически гистограмма строится следующим образом. Сначала множество значений, которое может принимать элемент выборки, разбивается на несколько интервалов. Чаще всего эти интервалы берут одинаковыми, но это не является строгим требованием. Эти интервалы откладываются на горизонтальной оси, затем над каждым рисуется прямоугольник. Если все интервалы были одинаковыми, то высота каждого прямоугольника пропорциональна числу элементов выборки, попадающих в соответствующий интервал. Если интервалы разные, то высота прямоугольника выбирается таким образом, чтобы его площадь была пропорциональна числу элементов выборки, которые попали в этот интервал.
	
	\subsubsection{Использование}
	Гистограммы применяются в основном для визуализации данных на начальном этапе статистической обработки. \newline Построение гистограмм используется для получения эмпирической оценки плотности распределения случайной величины. Для построения гистограммы наблюдаемый диапазон изменения случайной величины разбивается на несколько интервалов и подсчитывается доля от всех измерений, попавшая в каждый из интервалов. Величина каждой доли, отнесенная к величине интервала, принимается в качестве оценки значения плотности распределения на соответствующем интервале.
	
	\subsection{Вариационный ряд}
	Вариационным рядом называется последовательность элементов выборки, расположенных в неубывающем порядке. Одинаковые элементы повторяются.
	Запись вариационного ряда: $x_{(1)}, x_{(2)}, \ldots, x_{(n)}$.
	Элементы вариационного ряда $x_{(i)} (i = 1, 2, \ldots, n)$ называются порядковыми статистиками.
	
	\subsection{Выборочные числовые характеристики}
	С помощью выборки образуются её числовые характеристики. Это числовые характеристики дискретной случайной величины $X^{*}$, принимающей выборочные значения $x_{(1)}, x_{(2)}, \ldots, x_{(n)}$.
	
	\subsubsection{Характеристики положения}
	\begin{itemize}
		\item Выборочное среднее \begin{equation}
									 \overline{x} = \frac{1}{n}\sum_{i=1}^{n}{x_i} \eqno(8)
								\end{equation}
		\item Выборочная медиана \begin{equation}
								 	med x = \begin{cases}
											 	x_{(l+1)} &\text{$ n=2l+1$}\\
											 	\frac{x_{(l)} + x_{(l+1)}}{2} &\text{$ n=2l$}
								 			\end{cases} \eqno(9)
								 \end{equation}
		\item Полусумма экстремальных выборочных элементов \begin{equation}
														       z_R = \frac{x_{(1)} + x_{(n)}}{2} \eqno(10)
														   \end{equation}
		\item Полусумма квартилей \newline Выборочная квартиль $z_p$ порядка $p$ определяется формулой \begin{equation}
				 	z_p = \begin{cases}
		             	  	x_{([np]+1)} &\text{$np - $дробное}\\
		      			    x_{(np)}&\text{$np - $целое}
		      			  \end{cases}\eqno(11)
				 \end{equation}
				 Полусумма квартилей \begin{equation}
				 					 	z_Q = \frac{z_{1/4} + z_{3/4}}{2} \eqno(12)
				 					 \end{equation}
		\item Усечённое среднее\begin{equation}
							   		z_{tr} = \frac{1}{n-2r}\sum_{i=r+1}^{n-r}{x_{(i)}}, r\approx\frac{n}{4} \eqno(13)					   	
							   \end{equation}
	\end{itemize}

	\subsubsection{Характеристики рассеяния}
	Выборочная дисперсия
	\begin{equation}
		D = \frac{1}{n}\sum_{i=1}^{n}{(x_i-\overline{x})^2} \eqno(14)
	\end{equation}