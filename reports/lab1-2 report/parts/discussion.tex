\subsection{Гистограмма и график плотности распределения}
	По полученным графикам можно сделать вывод о том, что чем больше размер выборки, тем точнее гистограмма будет повторять график плотности вероятности того закона распределения, для которого была сгенерирована выборка. На маленьких выборках определить закон распределения почти невозможно.
\subsection{Характеристики положения и рассеяния}
	Исходя из данных, приведенных в таблицах, можно судить о том, что дисперсия характеристик рассеяния для распределения Коши является некой аномалией: значения слишком большие даже при увеличении размера выборки - понятно, что это результат выбросов, которые мы могли наблюдать в результатах предыдущего задания.