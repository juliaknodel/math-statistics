\subsection{Ядерные оценки плотности распределения}
	Для двумерного нормального распределения дисперсии выборочных коэффициентов корреляции упорядочены следующим образом: $r < r_{S} < r_{Q}$; для смеси распределений получили обратную картину: $r_{Q} < r_{S} < r$.
	\newline
	Процент попавших элементов выборки в эллипс рассеивания (95$\%$-ная доверительная область) примерно равен его теоретическому значению (95$\%$).
	
\subsection{Оценки коэффициентов линейной регрессии}
По полученным результатам (см. метрику удаленности модельной прямой от теоретической - distance) можно сказать, что используя критерий наименьших квадратов удастся точнее оценить коэффициенты линейной регрессии для выборки без возмущений. Если же редкие возмущения присутствуют, тогда лучше использовать критерий наименьших модулей.

\subsection{Проверка гипотезы о законе распределения генеральной совокупности. Метод хи-квадрат}
Заключаем, что гипотеза $H_{0}$ о нормальном законе распределения $N(x,\hat{\mu}, \hat{\sigma})$ на уровне значимости $\alpha = 0.05$ согласуется с выборкой для нормального распределения $N(x, 0, 1)$.
\newline
Также видно, что для выборок сгенерированных по закону Лапласа гипотеза $H_{0}$ оказалась принята.

\subsection{Доверительные интервалы для параметров распределения}
По полученным результатам видно, что параметры распределения $m = 0$ и $\sigma = 1$ лежат внутри доверительных интервалов. Также для выборки большего размера длина доверительных интервалов оказывается меньше, другими словами эти доверительные интервалы являются более точными.