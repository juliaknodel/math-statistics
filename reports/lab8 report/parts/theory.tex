\subsection{Доверительные интервалы для параметров нормального распределения}
\subsubsection{Доверительный интервал для математического ожидания $m$ нормального распределения}
Дана выборка ($x_{1},x_{2}, ... ,x_{n}$) объёма n из нормальной генеральной совокупности. На её основе строим выборочное среднее $\bar{x}$ и выборочное среднее квадратическое отклонение $s$. Параметры $m$ и $\sigma$ нормального распределения неизвестны.
\newline
Доказано, что случайная величина
\begin{equation}
T = \sqrt{n - 1}\cdot\frac{\bar{x} - m}{s}
\label{T}
\end{equation}
называемая статистикой Стьюдента, распределена по закону Стьюдента с $n-1$ степенями свободы. Пусть $f_{T}(x)$ — плотность вероятности этого распределения. Тогда 
\begin{multline}
P\left(-x < \sqrt{n - 1}\frac{\bar{x} - m}{s} < x \right) = 
P\left(-x < \sqrt{n - 1}\frac{m - \bar{x}}{s} < x \right) = \\\
= \int_{-x}^{x}{f_{T}(t)dt} = 2 \int_{0}^{x}{f_{T}(t)dt} = 
2\left(  \int_{-\infty}^{x}{f_{T}(t)dt} - \frac{1}{2} \right) = 2F_{T}(x) - 1
\label{P_f_t}
\end{multline}
Здесь $F_{T}(x)$ — функция распределения Стьюдента с $n-1$ степенями свободы.
\newline
Полагаем $2F_{T}(x)-1 = 1-\alpha$, где $\alpha$ — выбранный уровень значимости. Тогда $F_{T}(x) = 1-\alpha/2$. Пусть $t_{1-\alpha/2}(n-1)$ — квантиль распределения Стьюдента с $n-1$ степенями свободы и порядка $1-\alpha/2$. Из предыдущих равенств мы получаем 
\begin{equation}
\begin{split}
P\left(\bar{x} - \frac{sx}{\sqrt{n-1}} < m <  \bar{x} + \frac{sx}{\sqrt{n-1}}\right) = 2F_{T}(x) - 1 = 1 - \alpha, \\
P(\bar{x} - \frac{st_{1-\alpha/2}(n-1)}{\sqrt{n-1}} < m < \bar{x}\frac{st_{1-\alpha/2}(n-1)}{\sqrt{n-1}}) = 1 - \alpha
\end{split}
\end{equation}
что и даёт доверительный интервал для $m$ с доверительной вероятностью $\gamma = 1-\alpha$ [1, с. 457-458].

\subsubsection{Доверительный интервал для среднего квадратического отклонения $\sigma$ нормального распределения}
Дана выборка ($x_{1},x_{2}, ... ,x_{n}$) объёма n из нормальной генеральной совокупности. На её основе строим выборочную дисперсию $s^{2}$. Параметры $m$ и $\sigma$ нормального распределения неизвестны. Доказано, что случайная величина $ns^{2}/\sigma^{2}$ распределена по закону $\chi^{2}$ с $n-1$ степенями свободы.
\newline
Задаёмся уровнем значимости $\alpha$ и находим квантили $\chi^{2}_{\alpha/2}(n-1)$ и $\chi^{2}_{1-\alpha/2}(n-1)$.
\newline
Это значит, что 
\begin{equation}
\begin{split}
P\left(\chi^{2}(n-1) < \chi^{2}_{\alpha/2}(n-1)\right) = \alpha/2, \\
P\left(\chi^{2}(n-1) < \chi^{2}_{1-\alpha/2}(n-1)\right) = 1-\alpha/2
\end{split}
\end{equation}
Тогда
\begin{multline}
P\left(\chi^{2}_{\alpha/2}(n-1) < \chi^{2}(n-1) < \chi^{2}_{1-\alpha/2}(n-1)\right) = \\\
P\left(\chi^{2}(n-1) < \chi^{2}_{1-\alpha/2}(n-1)\right) -P\left(\chi^{2}(n-1) < \chi^{2}_{\alpha/2}(n-1)\right) = \\\ = 1 - \alpha/2 -\alpha/2 = 1 - \alpha
\label{P_chi_2}
\end{multline}
Отсюда
\begin{multline}
P\left(\chi^{2}_{\alpha/2}(n-1) < \frac{ns^{2}}{\sigma^{2}} < \chi^{2}_{1-\alpha/2}(n-1)\right) =
P\left(\frac{1}{\chi^{2}_{1-\alpha/2}(n-1)} < \frac{\sigma^{2}}{ns^{2}} < \frac{1}{\chi^{2}_{\alpha/2}(n-1)} \right) = \\\ =
P\left(\frac{s\sqrt{n}}{\sqrt{\chi^{2}_{1-\alpha/2}(n-1)}} < \sigma <  \frac{s\sqrt{n}}{\sqrt{\chi^{2}_{\alpha/2}(n-1)}}\right) = 1- \alpha
\label{interv}
\end{multline}
Окончательно
\begin{equation}
P\left(\frac{s\sqrt{n}}{\sqrt{\chi^{2}_{1-\alpha/2}(n-1)}} < \sigma <  \frac{s\sqrt{n}}{\sqrt{\chi^{2}_{\alpha/2}(n-1)}}\right) = 1- \alpha,
\label{fin_interval}
\end{equation}
что и даёт доверительный интервал для $\sigma$ с доверительной вероятностью $\gamma = 1 - \alpha$ [1, с. 458-459].

\subsection{Доверительные интервалы для математического ожидания $m$ и среднего квадратического отклонения $\sigma$ произвольного распределения при большом объёме выборки. Асимптотический подход}
При большом объёме выборки для построения доверительных интервалов может быть использован асимптотический метод на основе центральной предельной теоремы.
\subsubsection{Доверительный интервал для математического ожидания $m$ произвольной генеральной совокупности при большом объёме выборки}
Выборочное среднее $\bar{x} = \frac{1}{n}\sum_{i = 1}^{n}{x_{i}}$ при большом объёме выборки является суммой большого числа взаимно независимых одинаково распределённых случайных величин. Предполагаем, что исследуемое генеральное распределение имеет конечные математическое ожидание $m$ и дисперсию $\sigma^{2}$. Тогда в силу центральной предельной теоремы центрированная и нормированная случайная величина $(\bar{x} - M\bar{x}) / \sqrt{D\bar{x}} = \sqrt{n}·(\bar{x}-m)/\sigma$ распределена приблизительно нормально с параметрами 0 и 1. Пусть
\begin{equation}
\Phi(x) = \frac{1}{2\pi}\int_{-\infty}^{x}{e^{-t^{2}/2}dt}
\label{f_lapl}
\end{equation}
- функция Лапласа. Тогда
\begin{multline}
P\left(-x < \sqrt{n}\frac{\bar{x} - m}{\sigma} < x \right) = 
P\left(-x < \sqrt{n}\frac{m - \bar{x}}{\sigma} < x \right) \approx \\\
\approx \Phi(x) - \Phi(-x)=\Phi(x) - [1 - \Phi(x)] = 2\Phi(x) - 1
\label{P_PHI}
\end{multline}
Отсюда
\begin{equation}
P\left(\bar{x} - \frac{\sigma x}{\sqrt{n}} < m < \bar{x} - \frac{\sigma x}{\sqrt{n}} \right) \approx 2\Phi(x) - 1
\label{P_fin_PHI}
\end{equation}
Полагаем $2\Phi(x) - 1 = \gamma = 1 - \alpha$; тогда $\Phi(x) = 1 - \alpha/2$. Пусть $u_{1-\alpha/2}$ — квантиль нормального распределения N(0,1) порядка $1-\alpha/2$. Заменяя в равенстве (\ref{P_fin_PHI}) $\sigma$ на $s$, запишем его в виде
\begin{equation}
P\left(\bar{x} - \frac{su_{1-\alpha/2}}{\sqrt{n}} < m < \bar{x} - \frac{su_{1-\alpha/2}}{\sqrt{n}} \right) \approx \gamma,
\label{P_fin_u}
\end{equation}
что и даёт доверительный интервал для $m$ с доверительной вероятностью $\gamma = 1-\alpha$ [1, с. 460].

\subsubsection{Доверительный интервал для среднего квадратического отклонения $\sigma$ произвольной генеральной совокупности при большом объёме выборки}
Выборочная дисперсия $s^{2} = \sum_{i = 1}^{n}{\frac{(x_{i} - \bar{x})^{2}}{n}}$ при большом объёме выборки является суммой большого числа практически взаимно независимых случайных величин (имеется одна связь $\sum_{i=1}^{n}{x_{i}} = n\bar{x}$, которой при большом n можно пренебречь). Предполагаем, что исследуемая генеральная совокупность имеет конечные первые четыре момента.
\newline
В силу центральной предельной теоремы центрированная и нормированная случайная величина $(s^{2}-Ms^{2})/\sqrt{Ds^{2}}$ при большом объёме выборки n распределена приблизительно нормально с параметрами 0 и 1. Пусть $\Phi(x)$ — функция Лапласа (\ref{f_lapl}). Тогда
\begin{equation}
P\left(-x < \frac{s^{2}-Ms^{2}}{\sqrt{Ds^{2}}} < x\right)
\approx \Phi(x) - \Phi(-x)=\Phi(x) - [1 - \Phi(x)] = 2\Phi(x) - 1
\label{P_as_sigma}
\end{equation}

Положим $2\Phi(x)-1 = \gamma = 1-\alpha$. Тогда $\Phi(x) = 1-\alpha/2$. Пусть $u_{1-\alpha/2} - $ корень этого уравнения $-$ квантиль нормального распределения $N(0,1)$ порядка $1-\alpha/2$. Известно, что $Ms^{2} = \sigma^{2} -\frac{\sigma^{2}}{n} \approx \sigma^{2} \text{ и } Ds^{2} = \frac{\mu_{4} -\mu_{2}^{2}}{n} + o(\frac{1}{n}) \approx \frac{\mu_{4} -\mu_{2}^{2}}{n}$. Здесь $\mu_{k} -$ центральный момент k-го порядка генерального распределения; $\mu_2 = \sigma^2; \mu_4 = M[(x-Mx)^4];$
$o(\frac{1}{n}) -$ бесконечно малая высшего порядка, чем $1/n$, при $n\rightarrow \infty$. Итак, $Ds^{2} \approx \frac{\mu_{4} -\mu_{2}^{2}}{n}$. Отсюда


\begin{equation}
Ds^{2} \approx \frac{\sigma^{4}}{n}(\frac{\mu_{4}}{\sigma^{4}} - 1) = 
\frac{\sigma^{4}}{n}((\frac{\mu_{4}}{\sigma^{4}} - 3) + 2) = \frac{\sigma^{4}}{n}(E + 2) \approx \frac{\sigma^{4}}{n}(e + 2),
\label{Ds_2}
\end{equation}
где E = $\frac{\mu_{4}}{\sigma^{4}} - 3$ — эксцесс генерального распределения, e = $\frac{m_{4}}{s^{4}} - 3$ — выборочный эксцесс; $m_{4} = \frac{1}{n}\sum_{i =1}^{n}{(x_{i} - \bar{x})^{4}}$  — четвёртый выборочный центральный момент. Далее,
\begin{equation}
\sqrt{Ds^{2}} \approx \frac{\sigma^{2}}{\sqrt{n}}\sqrt{e + 2}
\label{sqrt_Ds}
\end{equation}
Преобразуем неравенства, стоящие под знаком вероятности в формуле
\newline
$P\left(-x < \frac{s^{2}-Ms^{2}}{\sqrt{Ds^{2}}} < x\right) = \gamma$:
\begin{equation}
\begin{split}
-\sigma^{2}U < s^{2} -\sigma^{2} < \sigma^{2}U; \\
\sigma^{2}(1-U) < s^{2} < \sigma^{2}(1 + U); \\
1/[\sigma^{2}(1 + U)] < 1/s^{2} < 1/[\sigma^{2}(1-U)];\\
s^{2}/(1 + U) < \sigma^{2} < s^{2}/(1-U);\\
s(1 + U)^{-1/2} < \sigma < s(1-U)^{-1/2},
\label{multi_ineq}
\end{split}
\end{equation}
где $U = u_{1-\alpha/2}\sqrt{(e+2)/n}$ или
\newline

$s(1 + u_{1-\alpha/2}\sqrt{(e + 2)/n})^{-1/2} < \sigma < s(1-u_{1-\alpha/2}\sqrt{(e + 2)/n})^{-1/2}$.
\newline
Разлагая функции в биномиальный ряд и оставляя первые два члена, получим
\begin{equation}
s(1-0.5U) < \sigma < s(1 + 0.5U)
\label{s_U}
\end{equation}
или
\begin{equation}
	s(1-0.5u_{1-\alpha/2}\sqrt{(e + 2)/n}) < \sigma < s(1 + 0.5 u_{1-\alpha/2}\sqrt{(e + 2)/n})
\label{s_u}
\end{equation}
Формулы (\ref{multi_ineq}) или (\ref{s_u}) дают доверительный интервал для $\sigma$ с доверительной вероятностью $\gamma = 1-\alpha$ [1, с. 461-462]. 
\newline
\textit{Замечание.} Вычисления по формуле (\ref{multi_ineq}) дают более надёжный результат, так как в ней меньше грубых приближений.