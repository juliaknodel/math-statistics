\subsection{Гистограмма и график плотности распределения}
По полученным графикам можно сделать вывод о том, что чем больше размер выборки, тем точнее гистограмма будет повторять график плотности вероятности того закона распределения, для которого была сгенерирована выборка. На маленьких выборках определить закон распределения почти невозможно.

\subsection{Характеристики положения и рассеяния}
Исходя из данных, приведенных в таблицах, можно судить о том, что дисперсия характеристик рассеяния для распределения Коши является некой аномалией: значения слишком большие даже при увеличении размера выборки - понятно, что это результат выбросов, которые мы могли наблюдать в результатах предыдущего задания.

\subsection{Боксплот Тьюки и доля выбросов}
По данным, приведенным в таблице, можно сказать, что чем больше выборка, тем ближе доля выбросов будет к теоретической оценке. Снова доля выбросов для распределения Коши значительно выше, чем для остальных распределений. Равномерное распределение же в точности повторяет теоретическую оценку - выбросов мы не получали. \\
\\
Боксплот Тьюки позволяет наглядно оценить наиболее важные характеристики распределений.

\subsection{Ядерные оценки плотности распределения}
	По полученным графикам, можно сказать, что чем больше размер выборки - тем точнее полученные по ней оценки будут аппроксимировать реальные генеральную функцию распределения и плотность вероятности. Хотя даже на выборке $n = 20$ результат получается очень близким к действительности и можно понять, что это за распределение.
	\\
	\\
	Также можно увидеть, что чем больше коэффициент при параметре сглаживания $\hat{h_n}$, тем меньше изменений знака производной у аппроксимирующей функции, вплоть до того, что при $h = 2h_n$ функция становится унимодальной на рассматриваемом промежутке. Также видно, что при $h = 2h_n$ по полученным приближениям становится сложно сказать плотность вероятности какого распределения они должны повторять, так как они очень похожи между собой.
	\\
	\\
	Заметим, что для распределения Лапласа лучше использовать б\'{о}льшие значения h, так как исчезают скачки, которых нет у функции плотности данного распределения и, наоборот, для равномерного лучше брать м\'{е}ньшие значения h, так как при б\'{о}льших получается, что плотность на концах промежутка меньше, чем в центре, что неверно. Именно при таком выборе значения $h$ ядерная оценка будет лучше воспроизводить особенности данных распределений.
	
	
	
	
	
	
	