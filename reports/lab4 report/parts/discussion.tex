\subsection{Ядерные оценки плотности распределения}
	По полученным графикам, можно сказать, что чем больше размер выборки - тем точнее полученные по ней оценки будут аппроксимировать реальные генеральную функцию распределения и плотность вероятности. Хотя даже на выборке $n = 20$ результат получается очень близким к действительности и можно понять, что это за распределение.
	
	Также можно увидеть, что чем больше коэффициент при параметре сглаживания $\hat{h_n}$, тем меньше изменений знака производной у аппроксимирующей функции, вплоть до того, что при $h = 2h_n$ функция становится унимодальной на рассматриваемом промежутке. Также видно, что при $h = 2h_n$ по полученным приближениям становится сложно сказать плотность вероятности какого распределения они должны повторять, так как они очень похожи между собой.