\subsection{Эмпирическая функция распределения}
	\subsubsection{Статистический ряд}
	Статистическим рядом называется последовательность различных элементов выборки z1,z2, ...,zk, расположенных в возрастающем порядке с указанием частот n1,n2, ...,nk, с которыми эти элементы содержатся в выборке. Статистический ряд обычно записывается в виде таблицы
	\begin{table}[H]
		\centering
		\begin{tabular}[t]{lrrrr}
			\hline
			z   &      $z_1$ & $z_2$ & ... & $z_k$ 		\\
			\hline
			n & $n_1$ & $n_2$ & ... & $n_k$   	\\
			\hline
		\end{tabular}
		\caption{Статистический ряд}
		\label{tab:normal}
	\end{table}

	\subsubsection{Определение}
	Эмпирической (выборочной) функцией распределения (э. ф. р.) называется
	относительная частота события $X < x$, полученная по данной выборке:
	\begin{equation}
		{F_n^*(x) = P^*(X<x)}
	\end{equation}
	
	
	\subsubsection{Описание}
	Для получения относительной частоты $P^*(X < x)$ просуммируем в ста-
	тистическом ряде, построенном по данной выборке, все частоты $n_i$, для
	которых элементы $z_i$ статистического ряда меньше $x$. Тогда 
	$P^*(X<x)=\frac{1}{n}\sum\limits_{z_i<x}n_i$. Получаем
	\begin{equation}
		{F^*(x)=\frac{1}{n}\sum\limits_{z_i<x}n_i}
	\end{equation}
	$F^*(x)$ — функция распределения дискретной случайной величины $X^*$, заданной таблицей распределения
	\begin{table}[H]
		\centering
		\begin{tabular}[t]{lrrrr}
			\hline
			$X^*$   &      $z_1$ & $z_2$ & ... & $z_k$ 		\\
			\hline
			P & $\frac{n_1}{n}$ & $\frac{n_2}{n}$ & ... & $\frac{n_k}{n}$   	\\
			\hline
		\end{tabular}
		\caption{Таблица распределения}
		\label{tab:normal}
	\end{table}
	Эмпирическая функция распределения является оценкой, т. е. приближённым значением, генеральной функции распределения
	\begin{equation}
		{F_n^*(x)\approx F_X(x)}
	\end{equation}
	
\subsection{Оценки плотности вероятности}
	\subsubsection{Определение}
		Оценкой плотности вероятности $f(x)$ называется функция $\hat{f}(x)$, построенная на основе выборки, приближенно равная $f(x)$
		\begin{equation}
			{\hat{f}(x) \approx f(x)}
		\end{equation}
	\subsubsection{Ядерные оценки}	
		Представим оценку в виде суммы с числом слагаемых, равным объёму выборки:
		\begin{equation}
			{\hat{f_n}(x) = \frac{1}{nh_n}\sum\limits_{i=1}^{n}K(\frac{x-x_i}{h_n})}
		\end{equation}
		Здесь функция $K(u)$, называемая ядерной (ядром), непрерывна и является
		плотностью вероятности, $x_1, ..., x_n$ — элементы выборки, ${h_n}$ — любая
		последовательность положительных чисел, обладающая свойствами
		\begin{equation}
			{h_n \xrightarrow[n\to\infty]{}0; \frac{h_n}{n^{-1}} \xrightarrow[n\to\infty]{}\infty}
		\end{equation}
		Такие оценки называются непрерывными ядерными.\\
		
		Замечание. Свойство, означающее сближение оценки с оцениваемой величиной при $n\to\infty$ в каком-либо смысле, называется состоятельностью оценки. \\
		Если плотность $f(x)$ кусочно-непрерывная, то ядерная оценка плотности
		является состоятельной при соблюдении условий, накладываемых на параметр сглаживания $h_n$, а также на ядро $K(u)$.
		Гауссово (нормальное) ядро
		\begin{equation}
			{K(u)=\frac{1}{\sqrt{2\pi}}e^{-\frac{u^2}{2}}}
		\end{equation}
		Правило Сильвермана
		\begin{equation}
			{h_n = 1.06\hat{\sigma}n^{-\frac{1}{5}}}
		\end{equation}
		где $\hat{\sigma}$ $-$ выборочное стандартное отклонение.