\subsection{Проверка гипотезы о законе распределения генеральной совокупности. Метод хи-квадрат}
Метод максимального правдоподобия:
\newline
$\hat{\mu} \approx 0.08, \hat{\sigma} \approx 1.02$
\newline
Критерий согласия $\chi^{2}$:
\newline
Количество промежутков k = 6
\newline
Уровень значимости $\alpha$= 0.05
\newline
Тогда квантиль $\chi^{2}_{1-\alpha}(k-1)$ = $\chi^{2}_{0.95}(5)$. Из таблицы [3, с. 358] $\chi^{2}_{0.95}(5) \approx 11.07$. 
\begin{table}[H]
	\centering
	\begin{tabular}{| c | c | c | c | c | c | c |}
		\hline
		$i$ & $limits$         &   $n_i$ &    $p_i$ &   $np_i$ &   $n_i - np_i$ &   $\frac{(n_i-np_i)^2}{np_i}$ \\
		\hline
		1 & [$-\infty$, -1.1] &    13 & 0.1357 &  13.57 &        -0.57 &                        0.02 \\
		2 & [-1.1, -0.55]  &    19 & 0.1555 &  15.55 &         3.45 &                        0.77 \\
		3 & [-0.55, 0.0]   &    14 & 0.2088 &  20.88 &        -6.88 &                        2.27 \\
		4 & [0.0, 0.55]    &    25 & 0.2088 &  20.88 &         4.12 &                        0.81 \\
		5 & [0.55, 1.1]    &    12 & 0.1555 &  15.55 &        -3.55 &                        0.81 \\
		6 & [1.1,  $\infty$]   &    17 & 0.1357 &  13.57 &         3.43 &                        0.87 \\
		$\sum$ & $-$              &   100 & 1      & 100    &        -0    &                        5.55 \\
		\hline
	\end{tabular}
	\caption{ Вычисление $\chi^{2}_{B}$ при проверке гипотезы $H_{0}$ о нормальном законе распределения $N(x,\hat{\mu}, \hat{\sigma})$}
	\label{tab:chi_2}
\end{table}
Сравнивая $\chi^{2}_{B} = 5.55$ и $\chi^{2}_{0.95}(5) \approx 11.07$, видим, что $\chi^{2}_{B}$ < $\chi^{2}_{0.95}(5)$.


	
