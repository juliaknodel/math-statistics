Для 5 распределений:
\begin{enumerate}
	\item $N(x, 0, 1)$ -- нормальное распределение
	\item $C(x, 0, 1)$ -- распределение Коши
	\item $L(x, 0, \frac{1}{\sqrt{2}})$ -- распределение Лапласа 
	\item $P(k, 10)$ -- распределение Пуассона
	\item $U(x, -\sqrt{3}, \sqrt{3})$ -- расномерное распределение
\end{enumerate}

\subsection{Задание 1}
Сгенерировать выборки размером 20 и 100 элементов.
Посмтроить для них боксплот Тьюки.
Для каждого распределения определить долю выбросов экспериментально (сгенерировав выборку. соттветствующую распределению 1000 раз, и вычислиы среднюю долю выбросов) и сравнить с результатами, полученными теоретически.
\subsection{Задание 2}
Сгенерировать выборки размером 20, 60 и 100 элементов.
Построить на них эмпирические функции распределения и ядерные оценки плотности распределения на отрезке [-4;4] для непрерывных распределений и на отрезке [6;14] для распределения Пуассона.
